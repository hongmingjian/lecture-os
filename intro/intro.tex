
%% 
%% $Id: intro.tex,v 1.18 2008/03/18 07:14:41 hmj Exp $
%% 
\documentclass[CJKutf8,dvipsnames,table]{beamer}
\usepackage{hyperref}
\hypersetup{
  pdftitle={Operating System Concepts},
  pdfauthor={Hong MingJian},
  pdfsubject={Introduction},
  pdfpagemode={FullScreen},
  colorlinks={true},
  linkcolor={blue},
}

%% https://tex.stackexchange.com/questions/84652/is-there-a-macro-telling-which-os-were-using
\usepackage{ifxetex,ifluatex}
\newif\ifxetexorluatex % a new conditional starts as false
\ifnum 0\ifxetex 1\fi\ifluatex 1\fi>0
   \xetexorluatextrue
\fi
\usepackage{ifplatform}
\ifxetexorluatex
	\usepackage[slantfont,boldfont]{xeCJK}
	\ifwindows
		\setCJKmainfont{SimSun} % Windows默认中文字体:中易宋体
	\fi
	\ifmacosx
		\setCJKmainfont{STSong} % MacOS默认中文字体:华文宋体
	\fi
	\iflinux
		\setCJKmainfont{Noto Serif CJK SC} % Linux默认中文字体:思源宋体(By Adobe & Google)
	\fi
\else
	\usepackage{CJKutf8}
\fi
	
\usetheme{Madrid}%{Warsaw}
\usecolortheme{crane}

%gets rid of bottom navigation bars
\setbeamertemplate{footline}[page number]{}
%gets rid of navigation symbols
\setbeamertemplate{navigation symbols}{}
	
\begin{document}
\ifxetexorluatex\else
\begin{CJK*}{UTF8}{song}  
\fi

  \title{操作系统原理}
  \subtitle{第一章:介 绍}
	\author{洪明坚}
	\institute{重庆大学软件学院}
  \date{\today}

  \AtBeginSection[]
  {
    \begin{frame}
      \frametitle{Outline}
      \tableofcontents[currentsection]
    \end{frame}
  }

  \frame{\titlepage}

  \frame{\frametitle{目录}\tableofcontents}

  \section{课程简介}
  \subsection{主要内容及参考资料}

  %% PAGE
  \begin{frame}
    \frametitle{课程简介} \pause
	  \begin{itemize}
	    \item 主要内容  \pause
	      \begin{itemize}
	      \item 操作系统的基本概念、原理和设计。  \pause
	      \end{itemize}
	    \item 教材及参考资料  \pause
	      \begin{itemize}
	      \item \href{https://www.os-book.com}{\textbf{Abraham Silberschatz, Operating System Concepts, 6th Edition, John Wiley \& Sons, Inc.}}  \pause
	        \begin{itemize}
	        \item 10th edition is also available.  \pause

	        \end{itemize}
	      \item \href{https://www.cs.vu.nl/~ast/books/mos2}{Andrew S. Tanenbaum, Modern operating System, 2nd Edition, Prentice Hall.}  \pause
	      \item Maurice J. Bach, The design of Unix operating system, Pearson education.  \pause
	      \end{itemize}
	    \begin{center}
	    	\includegraphics[scale=0.4]{oscv6}  \pause
			\hspace{1mm}
	    	\includegraphics[scale=0.2]{mosv2}  \pause
			\hspace{1mm}
	    	\includegraphics[scale=0.15]{uos}    \pause
	    \end{center}
%	    \item 作业  \pause
%	      \begin{itemize}
%	      \item 记入总成绩,严禁抄袭! 
%	      \end{itemize}
	  \end{itemize}
  \end{frame}

  \subsection{为什么要学习操作系统原理?}

  %% PAGE
  \begin{frame}
    \frametitle{为什么要学习操作系统原理?} \pause
	  \begin{itemize}
	  \item 操作系统是最重要的系统软件之一,  \pause
	    \begin{itemize}
	    \item 没有操作系统的计算机几乎不能使用;  \pause
	    \item 学习操作系统将帮助软件开发人员开发出优秀的应用软件系统。  \pause
	    \end{itemize}
	  \item 操作系统是最复杂的系统软件之一,  \pause
	    \begin{itemize}
	    \item 学习计算机系统是如何工作的;  \pause
	    \item 学习如何通过\textbf{抽象}来掌握复杂的软件系统;  \pause
	    \item 学习如何系统地设计复杂的软件系统。 
	    \end{itemize}
	  \end{itemize}
  \end{frame}

  \section{What's an Operating System?}
  \subsection{Components of a Computer System}

  %% PAGE
  \begin{frame}
    \frametitle{Components of a Computer System} \pause
	  \begin{center}
	    \includegraphics[scale=0.5]{v6f1-1}
	  \end{center}
  \end{frame}

  \subsection{What's an Operating System?}

  %% PAGE
  \begin{frame}
    \frametitle{What's an Operating System?} \pause
	  \begin{itemize}
	  \item Different views  \pause
	    \begin{itemize}
	    \item The operating system controls and coordinates the use of the hardware among the various application programs for the various users.  \pause
	      \begin{itemize}
	      \item \textbf{Control program}  \pause
	      \end{itemize}
	    \item The operating system manages the resources of a computer so that various applications and users can operate the computer system efficiently and fairly.  \pause
	      \begin{itemize}
	      \item \textbf{Resource manager (Resource allocator)}  \pause
	      \end{itemize}
	    \item The operating system abstracts the computer hardware and presents the user a friendly interface.  \pause
	      \begin{itemize}
	      \item \textbf{Extended machine (Virtual machine)}  \pause
	      \end{itemize}
	    \end{itemize}
	  \item So, what's an operating system?  \pause
	    \begin{itemize}
	    \item \textbf{No} common accepted definition is available.  \pause
	    \item The operating system exists because they are a reasonable way to solve the problem of creating a usable computing system. 
	    \end{itemize}
	  \end{itemize}
  \end{frame}

  \subsection{Components of an operating system}

  %% PAGE
  \begin{frame}
    \frametitle{Components of an operating system} \pause
	  \begin{itemize}
	  \item What's the part of the operating system?  \pause
	    \begin{itemize}
	    \item Various people, company or organization has different opinions.  \pause
	      \begin{itemize}
	      \item Microsoft insisted that web browser and media player were parts of the operating system.  \pause
	      \end{itemize}
	    \end{itemize}
	  \item In this course, we will focus on the \textbf{kernel} of the operating system.  \pause
	    \begin{itemize}
	    \item The kernel provides the lowest level of abstraction layer for the resources (especially memory, processors and I/O devices). It includes (but is not limited to) the following components  \pause
	      \begin{itemize}
	      \item \textbf{CPU manager}  \pause
	      \item \textbf{Memory Manager}  \pause
	      \item File System  \pause
	      \item Device Manager 
	      \end{itemize}
	    \end{itemize}
	  \end{itemize}
  \end{frame}

  \subsection{History of Operating System}

  %% PAGE
  \begin{frame}
    \frametitle{History of Operating System} \pause
	  \begin{itemize}
	  \item To see what operating systems are and what they do, we will consider how they have developed over past 45 years.  \pause
	  \item Operating system runs on specific hardwares.  \pause
	    \begin{itemize}
	    \item We cannot understand operating system without some knowledge of underlying hardware.  \pause
	    \end{itemize}
	  \item So, we will trace the evolution of computer systems and their operating systems to identify the common elements of operating systems.  \pause
	    \begin{itemize}
	    \item Mainframes and mini-computers  \pause
	      \begin{itemize}
	      \item Mainframe - IBM System z9  \pause
	      \item Mini-computer - IBM System i  \pause
	      \end{itemize}
	    \item Desktop computers  \pause
	      \begin{itemize}
	      \item Apple II, Macintosh  \pause
	      \item IBM Personal Computer  \pause
	      \end{itemize}
	    \item Embedded computers 
	    \end{itemize}
	  \end{itemize}
  \end{frame}

  %% PAGE
  \begin{frame}
    \frametitle{Operating systems for mainframes and minicomputers} \pause
	  \begin{itemize}
	  \item Mainframes and minicomputers usually have dedicated operating systems.  \pause
	    \begin{itemize}
	    \item \textbf{zOS} is the operating system for IBM System z9.  \pause
	    \item \textbf{OS/400} is the operating system for IBM System i.  \pause
	    \end{itemize}
	  \item They evolved from simple \textbf{batch system}, to \textbf{multiprogramming system} and to \textbf{time-sharing system}. 
	  \end{itemize}
  \end{frame}

  %% PAGE
  \begin{frame}
    \frametitle{Batch systems(1/4)} \pause
	  \begin{itemize}
	  \item The computer runs one and only one application at a time.  \pause
	  \item Batching similar jobs  \pause
	    \begin{itemize}
	    \item Automatically transfers control from one job to another.  \pause
	    \end{itemize}
	  \item First rudimentary operating system. 
	  \end{itemize}
  \end{frame}

  %% PAGE
  \begin{frame}
    \frametitle{Batch systems(2/4)} \pause
	  \begin{center}
	    \includegraphics[scale=0.4]{mosv2f1-2} \pause
	  \end{center}
	  \begin{itemize}
	  \item \scriptsize{(a) Programmers bring cards to IBM 1401} \pause \scriptsize{(b) 1401 reads batch of jobs into tape.}  \pause
	  \item \scriptsize{(c) Operator carries input tape to IBM 7094} \pause  \scriptsize{(d) 7094 does computing.}  \pause
	  \item \scriptsize{(e) Operator carries output tape to IBM 1401.} \pause \scriptsize{(f) 1401 prints output.} 
	  \end{itemize}
  \end{frame}

  %% PAGE
  \begin{frame}
    \frametitle{Batch systems(3/4)} \pause
	  \begin{itemize}
	  \item A typical job:   \pause
	  \end{itemize}
	  \begin{center}
	    \includegraphics[scale=0.5]{mosv2f1-3}
	  \end{center}
  \end{frame}

  %% PAGE
  \begin{frame}
    \frametitle{Batch system(4/4)} \pause
	  \begin{itemize}
	  \item A punch card  \pause
	  \end{itemize}
	  \begin{center}
	    \includegraphics[scale=0.5]{punchcard}
	  \end{center}
  \end{frame}

  %% PAGE
  \begin{frame}
    \frametitle{Multiprogramming systems} \pause
	  \begin{itemize}
	  \item Several jobs are kept in main memory at the same time, and the CPU is multiplexed among them.  \pause
	  \end{itemize}
	  \begin{center}
	    \includegraphics[scale=0.5]{v6f1-3}
	  \end{center}
  \end{frame}

  %% PAGE
  \begin{frame}
    \frametitle{Batch v.s. Multiprogramming system} \pause
	  \begin{center}
	    \includegraphics[scale=0.5]{multiprogramming}
	  \end{center}
  \end{frame}

  %% PAGE
  \begin{frame}
    \frametitle{Time-sharing systems} \pause
	  \begin{itemize}
	  \item The CPU is multiplexed among several jobs that are kept in memory and on disk (the CPU is allocated to a job only if the job is in memory).  \pause
	    \begin{itemize}
	    \item Designed for interactive computing which requires quick response time.  \pause
	    \end{itemize}
	  \end{itemize}
	  \setlength{\unitlength}{0.5cm}
	  \thicklines
	  \begin{picture}(22, 10)
	    \put(9, 1){\framebox(4, 1.5){CPU}}
	    \put(5, 4){\oval(3, 1.5)}\put(4.3, 3.8){Job 1}
	    \put(8, 6){\oval(3, 1.5)}\put(7.3, 5.8){Job 2}
	    \put(11, 8){\oval(3, 1.5)}\put(10.3, 7.8){Job 3}
	    \put(14, 6){\oval(3, 1.5)}\put(13.3, 5.8){Job 4}
	    \put(17, 4){\oval(3, 1.5)}\put(16.3, 3.8){Job 5}
	    \put(11, 4){\vector(4, 3){2}}
	  \end{picture}
	  \thinlines
  \end{frame}

  %% PAGE
  \begin{frame}
  \frametitle{Operating systems for desktop computers} \pause
	  \begin{itemize}
	  	\item Operating systems for these computers have benefited in several ways from the development of the operating systems for mainframes.  \pause
	    \begin{itemize}
		    \item Apple Macintosh  \pause
		    \item Microsoft MS-DOS, Windows 9x/NT  \pause
		    \item IBM OS/2  \pause
		    \item Unix  \pause
		      \begin{itemize}
		      \item \textbf{BSD} (Berkeley Software Distribution) Unix, including \href{https://www.freebsd.org/}{FreeBSD}, \href{https://www.openbsd.org/}{OpenBSD} and \href{https://www.netbsd.org/}{NetBSD}  \pause
		      \item Solaris by Sun microsystem (acquired by Oracle on 2009/04/20)  \pause
		      \item Apple macOS  \pause
		      \end{itemize}
			\item Linux  \pause
		    \item ...... 
	    \end{itemize}
	  \end{itemize}
  \end{frame}

  %% PAGE
  \begin{frame}
    \frametitle{Operating systems for embedded systems} \pause
	  \begin{itemize}
	  \item Embedded computer is often used as a control device in a dedicated application such as industrial control systems. Usually, they have limited resources:   \pause
	    \begin{itemize}
	    \item Slow processor, \pause limited memory.  \pause
	    \item Small or even no display screen.  \pause
	    \item Limited power supply, etc  \pause
	    \end{itemize}
	  \item Some control devices have time requirement, i.e., \textbf{Real time}  \pause
	    \begin{itemize}
	    \item \textbf{Hard} real time, actions absolutely MUST occur at a certain moment.  \pause
	    \item \textbf{Soft} real time, missing an occasional deadline is acceptable.  \pause
	    \end{itemize}
	  \item Examples  \pause
	    \begin{itemize}
	    %\item Microsoft Windows CE (Consumer Electronics)  \pause
	    \item Windriver: vxWorks  \pause
            \item BlackBerry: QNX  \pause
	    \item Linux  \pause
	    \item ...... 
	    \end{itemize}
	  \end{itemize}
  \end{frame}

  \subsection{Features migration}

  %% PAGE
  \begin{frame}
    \frametitle{Features migration} \pause
	  \begin{itemize}
	  \item History repeats itself.  \pause
	  \end{itemize}
	  \begin{center}
	    \includegraphics[scale=0.4]{v6f1-6}
	  \end{center}
  \end{frame}

  %% Questions
  \begin{frame}
    \frametitle{Questions}
	  \begin{itemize}
	  \item Any questions? 
	  \end{itemize}
	  \begin{center}
	    \includegraphics[scale=.5]{question}
	  \end{center}
  \end{frame}
  
  %% PAGE
\ifxetexorluatex\else
\end{CJK*}
\fi
\end{document}
