
%%
%% $Id: comparch.tex,v 1.13 2008/03/18 07:14:41 hmj Exp $
%%
\documentclass[CJKutf8,dvipsnames,table]{beamer}
\usepackage{hyperref}
\hypersetup{
  pdftitle={Operating System Concepts},
  pdfauthor={Hong MingJian},
  pdfsubject={Computer system structure},
  pdfpagemode={FullScreen},
  colorlinks={true},
  linkcolor={blue},
}

%% https://tex.stackexchange.com/questions/84652/is-there-a-macro-telling-which-os-were-using
\usepackage{ifxetex,ifluatex}
\newif\ifxetexorluatex % a new conditional starts as false
\ifnum 0\ifxetex 1\fi\ifluatex 1\fi>0
   \xetexorluatextrue
\fi

\ifxetexorluatex
	\usepackage[slantfont,boldfont]{xeCJK}
	\usepackage{ifplatform}
	\ifwindows
		\setCJKmainfont{SimSun} % Windows默认中文字体:中易宋体
	\fi
	\ifmacosx
		\setCJKmainfont{STSong} % MacOS默认中文字体:华文宋体
	\fi
	\iflinux
		\setCJKmainfont{Noto Serif CJK SC} % Linux默认中文字体:思源宋体(By Adobe & Google)
	\fi
\else
	\usepackage{CJKutf8}
\fi

\usepackage{listings}
\lstset{
  language=[ANSI]C,
  basicstyle=\scriptsize,
  tabsize=2,
  breaklines=true,
  keywordstyle=\color{blue},
  identifierstyle=,
  commentstyle=\color{OliveGreen},
  stringstyle=,
  showstringspaces=false,
  extendedchars=false
  % numbers=left,
  % numberstyle=\tiny
}

\usetheme{Madrid}%{Warsaw}
\usecolortheme{crane}

%gets rid of bottom navigation bars
\setbeamertemplate{footline}[page number]{}
%gets rid of navigation symbols
\setbeamertemplate{navigation symbols}{}

\begin{document}
\ifxetexorluatex\else
\begin{CJK*}{UTF8}{song}
\fi

  \title{操作系统原理}
  \subtitle{第二章:计算机系统结构}
  \author{洪明坚}
  \institute{重庆大学软件学院}
  \date{\today}

  % \AtBeginSection[]
  % {
  %   \begin{frame}
  %     \frametitle{Outline}
  %     \tableofcontents[currentsection]
  %   \end{frame}
  % }

  \frame{\titlepage}

  \frame{\frametitle{目录}\tableofcontents}

  \section{Computer system structure}

  %% PAGE
  \begin{frame}
    \frametitle{Computer system structure} \pause
    \begin{itemize}
    \item We need to have a general knowledge of the structure of a computer system before we can explore the details of the operating system.  \pause
      \begin{itemize}
      \item You should be familiar with these concepts in the course \emph{Computer architecture}.   \pause
      \item But, some of them will be explored in-depth with the operating system in mind.  \pause
      \end{itemize}
    \end{itemize}
  \end{frame}

  %% PAGE
  \begin{frame}
    \frametitle{Overall structure} \pause
    \begin{center}
      \includegraphics[scale=.35]{v6f2-1} \pause
    \end{center}
    \begin{itemize}
    \item The CPU and device controllers can execute concurrently, competing for memory cycles.  \pause
    \item To ensure orderly access to the memory, a memory controller is provided to synchronize access to the memory. 
    \end{itemize}
  \end{frame}

  \subsection{Bootstrap}

  %% PAGE
  \begin{frame}
    \frametitle{Bootstrap} \pause
    \begin{itemize}
    \item When users just power on a computer, there is no already running operating system available.  \pause
      \begin{itemize}
      \item We must load the operating system kernel from some persistent storages, such as disk and network server, to memory.  \pause
      \item Then the control is transfered to the entry of the operating system with a very basic environment.  \pause
      \item This procedure is named \textbf{bootstrap} (or simply \textbf{boot}) an operating system.  \pause
      \item The program which boots the operating system is called \textbf{boot-loader}.  \pause
      \end{itemize}
    \item Examples 
      \begin{itemize}
      \item \textbf{NTLDR} - boot-loader for Windows NT/2000/XP (resides in C:$\backslash$).  \pause
      \item \textbf{BOOTMGR} - boot-loader for Windows Vista/7/8/10 (resides in C:$\backslash$).  \pause
	      \begin{itemize}
          \item Loads the C:$\backslash$Windows$\backslash$System32$\backslash$winload.exe who finally loads the kernel.   \pause
	      \end{itemize}
      \item \textbf{GRUB} - one of boot-loaders for the Unix/Linux.  \pause
        % \item Question: Who loads the boot-loader?  \pause
      \end{itemize}
    %\item Bear in mind that  \pause
    %  \begin{itemize}
    %  \item \textbf{The boot-loader is NOT part of the operating system.} 
    %  \end{itemize}
    \end{itemize}
  \end{frame}

  %% PAGE
  \begin{frame}
    \frametitle{Questions}
    \begin{itemize}
    \item Any questions? 
    \end{itemize}
    \begin{center}
      \includegraphics[scale=.5]{question}
    \end{center}
  \end{frame}

  \subsection{Interrupt and Exception}

  %% PAGE
  \begin{frame}
    \frametitle{Interrupt} \pause
    \begin{itemize}
    \item Modern computers and operating systems are \textbf{interrupt driven}.  \pause
      \begin{itemize}
        % \item If there is nothing to do, the operating system will sit quietly, waiting for something to happen.  \pause
      \item Peripheral devices use \textbf{interrupt} to signal the CPU that something has happened.  \pause
      \end{itemize}
    \item When the CPU is interrupted, it must \textbf{serve the interrupt} by  \pause
      \begin{enumerate}
      \item \emph{Hardware}: saves some of registers and branches to the \textbf{interrupt service routine (ISR)};  \pause
      \item \emph{Assembly language procedure in ISR}: saves rest of registers if necessary and sets up a convenient environment;  \pause
      \item \emph{C language procedure in ISR}: does serve the interrupt, typically reads and buffers input data from peripheral device;  \pause
      \item \emph{C language procedure in ISR}: returns to the \emph{assembly language procedure in ISR};  \pause
      \item \emph{Assembly language procedure in ISR}: restores saved registers and returns to the location being interrupted.  
      \end{enumerate}
    \end{itemize}
  \end{frame}

  %% PAGE
  \begin{frame}
    \frametitle{Interrupt Vector} \pause
    \begin{itemize}
    \item Usually, a computer system has several peripheral devices.  \pause
    \item When an interrupt occurs, CPU must know which device triggered it.  \pause
      \begin{itemize}
      \item Computer system assigns each device an \emph{unique} interrupt request number (e.g., an 8-bit integer), or simply \textbf{IRQ}.  \pause
      \item The addresses of all ISRs are collected into a table called \textbf{interrupt vector table}, or simply \textbf{IVT}.  \pause
      \item When servicing an interrupt, CPU uses IRQ to index the interrupt vector to fetch the address of ISR and branches to it. 
      \end{itemize}
    \end{itemize}
  \end{frame}

  %% PAGE
  \begin{frame}
    \frametitle{Put them all together} \pause
    \begin{center}
      \includegraphics[scale=0.5]{isr}
    \end{center}
  \end{frame}

  %% PAGE
  \begin{frame}
    \frametitle{Exception} \pause
    \begin{itemize}
    \item Interrupt  \pause
      \begin{itemize}
      \item Triggered by peripheral devices;  \pause
      \item Asynchronous.  \pause
      \end{itemize}
    \item \textbf{Exception}  \pause
      \begin{itemize}
        % \item Triggered by an error (such as division by zero and invalid memory access) \pause or by a specific request from a user program that an operating system service be performed.  \pause
      \item Exceptions occur when the processor detects an error condition while executing an instruction, such as division by zero and invalid memory access.  \pause
      \item Synchronous.  \pause
      \end{itemize}
    \item Other than the above, handling of interrupts and exceptions is identical.  \pause
      \begin{itemize}
      \item Exception is also known as \textbf{software-generated interrupt 
        or \textbf{synchronous interrupt}.}
      \end{itemize}
    \end{itemize}
  \end{frame}

  %% PAGE
  \begin{frame}
    \frametitle{Questions}
    \begin{itemize}
    \item Any questions? 
    \end{itemize}
    \begin{center}
      \includegraphics[scale=.5]{question}
    \end{center}
  \end{frame}

  \subsection{I/O structure}

\iffalse

  %% PAGE
  \begin{frame}
    \frametitle{I/O structure} \pause
    \begin{itemize}
    \item When CPU is doing I/O with a peripheral device, two methods are available:  \pause
      \begin{itemize}
      \item (a) synchronous \footnote{Also known as blocking or non-overlapping I/O.} \pause and (b) asynchronous \footnote{Also known as non-blocking or overlapping I/O.} I/O.  \pause
      \end{itemize}
    \end{itemize}
    \begin{center}
      \includegraphics[scale=0.4]{v6f2-3}
    \end{center}
  \end{frame}

  %% PAGE
  \begin{frame}
    \frametitle{DMA (1/2)} \pause
    \begin{itemize}
    \item Asynchronous I/O requires one interrupt per byte/word.  \pause
      \begin{itemize}
      \item Even ISR is highly optimized, it does burn hundreds to thousands of CPU cycles.  \pause
      \item This is unacceptable for high-speed devices, such as disk and network adapter.  \pause
      \end{itemize}
    \item Here comes the DMA (Direct Memory Access).  \pause
      \begin{itemize}
      \item After setting up addresses and counters for I/O device, the device controller transfers the entire block of data directly to or from its own buffer to memory, with no intervention by the CPU.   \pause
      \item The DMA controller interrupts the CPU when the transfer has been completed. 
      \end{itemize}
    \end{itemize}
  \end{frame}

  %% PAGE
  \begin{frame}
    \frametitle{DMA(2/2)} \pause
    \begin{center}
      \includegraphics[scale=0.5]{mosv2f5-4}
    \end{center}
  \end{frame}

\fi

  %% PAGE
%  \begin{frame}
%    \frametitle{Example} \pause
%    \begin{center}
%      \includegraphics[scale=0.5]{v6f2-2}
%    \end{center}
%  \end{frame}

  %% PAGE
  \begin{frame}
    \frametitle{How does CPU access peripheral devices?} \pause
    \begin{itemize}
    \item CPU accesses devices through device controllers.  \pause
      \begin{itemize}
      \item Device controllers include registers to hold commands and the data being transferred.  \pause
      \item How does CPU access these registers?  \pause
      \end{itemize}
    \item Two methods:  \pause
      \begin{itemize}
      \item I/O port  \pause
      \item Memory-mapped I/O 
      \end{itemize}
    \end{itemize}
  \end{frame}

  %% PAGE
  \begin{frame}
    \frametitle{I/O port (1/2)} \pause
    \begin{itemize}
    \item All registers within device controllers are collected.  \pause
      \begin{itemize}
      \item An unique address (which is called \textbf{port}, an 8- or 16-bit integer) is assigned to each of them.  \pause
      \end{itemize}
    \item Special I/O instructions are designed to allow data transfers between these registers and memory.  \pause
    \item Example: IBM-PC  \pause
      \begin{itemize}
      \item 16-bit I/O ports are used to address the registers of device controllers.  \pause
      \item Two special I/O instructions: \textbf{IN} and \textbf{OUT} are included in the INTEL x86 CPU.  \pause
        \begin{itemize}
        \item IN \emph{reg}, \emph{port} - Read a byte/word from \emph{port} to CPU register \emph{reg}.  \pause
        \item OUT \emph{port}, \emph{reg} - Write the content of CPU register \emph{reg} to \emph{port}. 
        \end{itemize}
      \end{itemize}
    \end{itemize}
  \end{frame}

  %% PAGE
  \begin{frame}
    \frametitle{I/O port (2/2)} \pause
    \begin{itemize}
    \item Part of PC I/O port address map  \pause
    \end{itemize}
    \small
    \begin{center}
      \rowcolors[]{1}{blue!20}{blue!10}
      \begin{tabular}{ll} \hline
        \textbf{Range (hex)} & \textbf{Function}\\[0.5ex] \hline\hline
        000-01F & 1st DMA controller\\ \hline
        020-03F & 1st Programmable Interrupt Controller (PIC)\\ \hline
        040-05F & Programmable Interval Timer (System timer)\\ \hline
        060-06F & Keyboard\\ \hline
        220-233 & Sound card\\ \hline
        3D0-3DF & Color Graphics Adapter\\ \hline
      \end{tabular} \pause
    \end{center}
    \normalsize
    \begin{itemize}
    \item Example: system timer  \pause
    \end{itemize}
    \begin{center}
      \includegraphics[scale=0.5]{ioportpit}
    \end{center}
  \end{frame}

  %% PAGE
  \begin{frame}
    \frametitle{Memory-mapped I/O (1/3)} \pause
    \begin{itemize}
    \item In the method of I/O port,  \pause
      \begin{itemize}
      \item we can view I/O ports as another separate address space, independent of memory address space.  \pause
      \end{itemize}
    \item Registers within device controller is just a piece of storage.  \pause
      \begin{itemize}
      \item Why not access these registers using the same method as memory?  \pause
      \item In this case, a (unique) \textbf{memory address} is assigned to every register,\textbf{NOT a port address}.  \pause
      \end{itemize}
    \item That's the Memory-Mapped I/O(MMIO).  \pause
      \begin{itemize}
      \item Memory-mapped I/O uses the same bus to address both memory and I/O devices  \pause
      \item In order to accommodate the I/O devices, areas of CPU addressable space must be reserved for I/O rather than memory. 
      \end{itemize}
    \end{itemize}
  \end{frame}

  %% PAGE
  \begin{frame}
    \frametitle{Memory-mapped I/O (2/3)} \pause
    \begin{itemize}
    \item Example: video controller of IBM-PC  \pause
      \begin{itemize}
      \item Each location on the screen is mapped to a memory location.  \pause
      \end{itemize}
    \end{itemize}
    \begin{center}
      \includegraphics[scale=0.5]{ioportdisp}
    \end{center}
  \end{frame}

  %% PAGE
  \begin{frame}
    \frametitle{Memory-mapped I/O (3/3)} \pause
    \begin{itemize}
    \item Advantages  \pause
      \begin{itemize}
      \item Every instruction that can reference memory can also reference device controller registers.  \pause
        \begin{itemize}
        \item The device drivers can be written entirely in C.  \pause
        \end{itemize}
      \item No special protection mechanism is needed to keep user processes from performing I/O.  \pause
      \end{itemize}
    \item Disadvantages  \pause
      \begin{itemize}
      \item Most computers nowadays have some form of caching memory words. But, \pause caching a device controller register would be disastrous.  \pause
      \end{itemize}
    \item Modern computer systems use both of them,  \pause
      \begin{itemize}
      \item with memory-mapped I/O for data buffers and separate I/O ports for the command registers,  \pause
      \item as in the previous example of \emph{Mobility Radeon 7500}. 
      \end{itemize}
    \end{itemize}
  \end{frame}

  %% PAGE
  \begin{frame}
    \frametitle{Questions}
    \begin{itemize}
    \item Any questions? 
    \end{itemize}
    \begin{center}
      \includegraphics[scale=.5]{question}
    \end{center}
  \end{frame}

\iffalse

  \subsection{Storage structure}

  %% PAGE
  \begin{frame}
    \frametitle{Storage structure} \pause
    \begin{center}
      \includegraphics[scale=.5]{mosv2f1-7}\\
      Storage hierarchy \pause
    \end{center}
    \begin{itemize}
    \item Computer program must be in main memory to be executed.  \pause
      \begin{itemize}
      \item RAM is \textbf{volatile}, means it will lose its contents when the power to the device is removed.  \pause
      \end{itemize}
    \item Data or program must be written to \textbf{nonvolatile} storage to be \textbf{persistent}. 
    \end{itemize}
  \end{frame}

  %% PAGE
  \begin{frame}
    \frametitle{Magnetic Disks} \pause
    \begin{center}
      \includegraphics[scale=.3]{v6f2-5} \pause
    \end{center}
    \begin{itemize}
    \item Every sector (also known as \textbf{disk block}) can be addressed by \textbf{(Cylinder, Head, Sector)}, or simply \textbf{CHS}.  \pause
      \begin{itemize}
      \item sizeof(sector) = 512 bytes since 30 years ago \footnote{IDEMA recommended increasing the sector size from 512 bytes to 4096 in 2006.}.  \pause
      \end{itemize}
    \end{itemize}
  \end{frame}

  %% PAGE
  \begin{frame}
    \frametitle{Cache structure} \pause
    \begin{center}
      \includegraphics[scale=0.4]{csappv1f6-25}
    \end{center}
  \end{frame}

  %% PAGE
  \begin{frame}
    \frametitle{Accessing cache} \pause
    \begin{center}
      \includegraphics[scale=0.5]{csappv1f6-33}
    \end{center}
  \end{frame}

  %% PAGE
  \begin{frame}
    \frametitle{Cache types} \pause
    \begin{itemize}
    \item Cache may be classified by different values of \textbf{E}, i.e., lines per set.  \pause
      \begin{itemize}
      \item E = 1, \textbf{direct-mapped cache;}  \pause
      \item $C/B>E>1$, \textbf{set associative cache};  \pause
        \begin{itemize}
        \item \emph{E}-way associative cache  \pause
        \end{itemize}
      \item $C/B=E$, \textbf{full associative cache}.  \pause
      \end{itemize}
      % \item Example: Output of \emph{x86info}\footnote{A program used to dump IA-32 CPU detail information for Unix/Linux.} on a INTEL Pentium 4  \pause
      \begin{example}[Output of \emph{x86info} on a INTEL Pentium 4]
        \begin{quote}
          \small
          \textbf{Processor name string}: Intel(R) Pentium(R) 4 CPU 3.00GH\\
          \textbf{Cache info}\\
          Instruction trace cache: 12K uOps, 8-way associative.\\
          L1 Data cache: 16KB, sectored, 8-way associative. 64 byte line size.\\
          L2 unified cache: 2MB, sectored, 8-way associative. 64 byte line size.\\
          \textbf{TLB info}\\
          Instruction TLB: 4K, 2MB or 4MB pages, fully associative, 64 entries.\\
          Data TLB: 4KB or 4MB pages, fully associative, 64 entries.\\
          \normalsize
        \end{quote}
      \end{example}
    \end{itemize}
  \end{frame}

\fi

  %% PAGE
  \begin{frame}
    \frametitle{Questions}
    \begin{itemize}
    \item Any questions? 
    \end{itemize}
    \begin{center}
      \includegraphics[scale=.5]{question}
    \end{center}
  \end{frame}

  \subsection{Hardware protection}

  %% PAGE
  \begin{frame}
    \frametitle{Hardware protection} \pause
    \begin{itemize}
    \item To ensure proper operation, we must protect the operating system and all other program and their data from any malfunctioning program.  \pause
    \item Hardware protection can be broken down different ways  \pause
      \begin{itemize}
      \item \textbf{Dual-mode operation} \pause - Prevent user programs taking over part of the OS and using this to overwrite other programs or even modify the OS itself.  \pause
      \item \textbf{Privileged instructions} \pause - Prevent user programs disrupting the normal operation of the system by issuing illegal I/O instructions.  \pause
      \item \textbf{Memory protection} \pause - Prevent a user program directly accessing the memory of another user program  or even operating system.  \pause
      \item \textbf{CPU protection} \pause - Prevent a user program from getting stuck in an infinite loop and never returning control to the operating system. 
      \end{itemize}
    \end{itemize}
  \end{frame}

  %% PAGE
  \begin{frame}
    \frametitle{Dual-mode operation} \pause
    \begin{itemize}
    \item We need at least two separate modes of operation:  \pause
      \begin{itemize}
      \item \textbf{User mode} \pause - Execution on behalf of user programs;  \pause
      \item \textbf{Monitor mode} \pause - Execution on behalf of operating system.  \pause
        \begin{itemize}
        \item Also known as \textbf{supervisor, system, privileged or kernel mode}.  \pause
        \end{itemize}
      \end{itemize}
    \item \textbf{Mode bit} is added to computer hardware to indicate the current mode:  monitor (0) or user (1).  \pause
      \begin{itemize}
      \item It's set to \emph{monitor} at system boot time. \pause The operating system is then loaded, and starts user programs in user mode.  \pause
      \end{itemize}
    \item When an interrupt or exception occurs hardware switches to monitor mode.  \pause
      \begin{itemize}
      \item Whenever the operating system gains control of the computer, it's in monitor mode.  \pause
      \item And the system always switches to user mode before passing control to a user program. 
      \end{itemize}
    \end{itemize}
  \end{frame}

  %% PAGE
  \begin{frame}
    \frametitle{Example} \pause
    \begin{itemize}
    \item INTEL IA-32 supports 4 modes to operate, named \textbf{protection rings}.  \pause
    \end{itemize}
    \begin{center}
      \includegraphics[scale=0.3]{x86rings} \pause
    \end{center}
    \begin{itemize}
    \item But, most operating systems running on IA-32 only use 2 of 4.  \pause
      \begin{itemize}
      \item Ring 0 as monitor mode; \pause ring 3 as user mode. 
      \end{itemize}
    \end{itemize}
  \end{frame}

   %% PAGE
  \begin{frame}
    \frametitle{Example (cont'd)} \pause
    \begin{itemize}
    \item \textbf{Mode bit} of INTEL IA-32  \pause
    \end{itemize}
    \begin{center}
      \setlength{\unitlength}{.5cm}
      \begin{picture}(20, 14)
        \color{red}
        \put(0, 0){\includegraphics[scale=.35]{x86eflags}} \pause
        \put(0, 6.6){\framebox(6, 0.45){}}
      \end{picture}
    \end{center}
  \end{frame}

  %% PAGE
  \begin{frame}
    \frametitle{I/O protection} \pause
    \begin{itemize}
    \item All I/O instructions are privileged instructions.  \pause
      \begin{itemize}
      \item The hardware allows privileged instructions to be executed only in monitor mode.  \pause
      \item{If these instructions are to be executed in user mode, the
        hardware does not execute the instruction, but rather treats it as
      illegal and generates an exception.} \pause
      \item For example, \textbf{IN} and \textbf{OUT} are 2 privileged instructions in INTEL IA-32.  \pause
      \end{itemize}
    \item Must ensure that a user program could never gain control of the computer in monitor mode.  
    \end{itemize}
  \end{frame}

  %% PAGE
  \begin{frame}
    \frametitle{Memory protection} \pause
    \begin{itemize}
    \item In order to have memory protection, add two registers that determine the range of legal addresses a program may access:  \pause
      \begin{itemize}
      \item \textbf{Base register} \pause - holds the smallest legal physical memory address;  \pause
      \item \textbf{Limit register} \pause - contains the size of the range.  \pause
      \end{itemize}
      % \item Memory outside the defined range is protected.  \pause
    \end{itemize}
    \begin{center}
      \includegraphics[scale=0.4]{v6f2-10} \pause
    \end{center}
    \begin{itemize}
    \item We will return to this topic when entering \emph{memory management}. 
    \end{itemize}
  \end{frame}

  %% PAGE
  \begin{frame}
    \frametitle{Example} \pause
    \begin{center}
      \includegraphics[scale=0.5]{v6f2-9}
    \end{center}
  \end{frame}

  %% PAGE
  \begin{frame}
    \frametitle{CPU protection} \pause
    \begin{itemize}
    \item The operating system can enforce policies only if it gets a chance to run.  \pause
      \begin{itemize}
      \item If a malfunctioning program entered an infinite loop and never returns control to the operating system, then CPU was out of the control from the operating system.  \pause
      \end{itemize}
    \item \textbf{Timer} \pause - interrupts CPU after a specified period to ensure operating system maintains control.  \pause
      \begin{itemize}
      \item Remember that when interrupt occurs, the operating system will get the control via ISR.  \pause
      \end{itemize}
%    \item The timer can also be used to implement time sharing. 
    \end{itemize}
  \end{frame}

  %% PAGE
  \begin{frame}
    \frametitle{Timer} \pause
    \begin{center}
      \includegraphics[scale=.2]{mosv2f5-31} \pause
    \end{center}
    \begin{itemize}
    \item The \emph{counter register} is decremented by 1 at each pulse.  \pause
      \begin{itemize}
      \item When the \emph{counter register} reaches zero, the timer will interrupt CPU.  \pause
      \item Then the \emph{counter register} will be reloaded with the value of \emph{holding register} and decrementing repeats.  \pause
      \end{itemize}
    \item Example: Timer in IBM-PC  \pause
      \begin{itemize}
      \item An INTEL i8253 programmable interval timer with 16-bit counter and holding registers and pulses reach at 1193182Hz. 
      \end{itemize}
    \end{itemize}
    % \begin{center}
    %   \includegraphics[scale=0.5]{ioportpit}
    % \end{center}

  \end{frame}

  %% PAGE
  \begin{frame}
    \frametitle{Questions}
    \begin{itemize}
    \item Any questions? 
    \end{itemize}
    \begin{center}
      \includegraphics[scale=.5]{question}
    \end{center}
  \end{frame}

  \subsection{System call}

  %% PAGE
  \begin{frame}
    \frametitle{System call} \pause
    \begin{itemize}
    \item The operating system does nothing useful itself.  \pause
      \begin{itemize}
      \item But it provides some useful services to the user programs, such as reading a file from disk and sending data to remote host via network adapter.  \pause
      \item How does operating system provide these services?  \pause
      \end{itemize}
    \item It's the \textbf{system call} \pause - the (well-defined) \textbf{interface} between the operating system and the user programs.  \pause
      \begin{itemize}
      \item User programs can \textbf{ONLY} request services provided by the operating system via system call.  \pause
      \item The system calls in the interface vary from operating system to operating system.  \pause
      \item Also known as \textbf{supervisor call}. 
      \end{itemize}
      % \item We will explain system call in detail by an example. 
    \end{itemize}
  \end{frame}

  %% PAGE
  \begin{frame}
    \frametitle{Example} \pause
    \begin{itemize}
    \item Most operating systems provide the service that can read some bytes from a file: \emph{read}  \pause
      \begin{itemize}
      \item \emph{count = read(fd, buffer, nbytes);}  \pause
      \item This system call reads \emph{nbytes} data from file \emph{fd} to the specified \emph{buffer} and returns the number of bytes actually read in \emph{count}. 
      \end{itemize}
      % \item You may treat it as a \textbf{library call} as usual.  \pause
      %   \begin{itemize}
      %   \item{But there is a sharp difference: \textbf{the system calls will
        %   trap into the operating system, while library calls do NOT}. } \pause
      %   \end{itemize}
    \end{itemize}
  \end{frame}

  %% PAGE
  \begin{frame}
    \frametitle{Procedure (1/2)} \pause
    \small
    \begin{minipage}[c]{0.45\textwidth}
      \begin{enumerate}
      \item[1-3]{Prepare parameters;} \pause
      \item[4]{Call the wrapper (written in assembly language) of system call;} \pause
      \item[5]{Store the \textbf{system call number} of \emph{read} into a register;} \pause
      \item[6]{Trap into the operating system;} \pause
      \item[7]{Get \textbf{system call service routine} for \emph{read} by indexing a \textbf{system call table} using system call number;} \pause
      \item[8-11]{System call service routine runs and returns to user programs on completion.} \pause
      \end{enumerate}
    \end{minipage}%
    \begin{minipage}[c]{0.55\textwidth}
      \begin{center}
        \includegraphics[scale=0.45]{mosv2f1-17}
      \end{center}
    \end{minipage}
    \normalsize
  \end{frame}

  %% PAGE
  \begin{frame}
    \frametitle{Procedure (2/2)} \pause
    \begin{center}
      \includegraphics[scale=.4]{v6f2-8} \pause
    \end{center}
    \begin{itemize}
    \item Resident monitor (or simply monitor) here means the operating system and  \pause
    \item \emph{n} is the system call number. 
    \end{itemize}
  \end{frame}

  %% PAGE
  \begin{frame}
    \frametitle{Trap into the operating system (1/2)} \pause
    \begin{itemize}
    \item The user programs cannot trap into the operating system directly.  \pause
    \item How to trap into the operating system?  \pause
      \begin{itemize}
      \item Method 1: \textbf{Exception (software-generated interrupt)}.  \pause
      \item Method 2: Special instruction. 
      \end{itemize}
    \end{itemize}
  \end{frame}

  %% PAGE
  \begin{frame}
    \frametitle{Trap into the operating system (2/2)} \pause
    \begin{itemize}
    \item Exception  \pause
      \begin{itemize}
      \item INTEL IA-32 provides an instruction to trigger a exception, \emph{INT}.  \pause
        \begin{itemize}
        \item For example, FreeBSD/Linux uses \emph{INT 0x80} to trap into the operating system and Windows NT/XP uses \emph{INT 0x2e}.  \pause
        \end{itemize}
      \end{itemize}
    \item Special instruction  \pause
      \begin{itemize}
      \item In addition, INTEL IA-32 provides two special instructions to trap into the operating system: \emph{SYSENTER and SYSEXIT} because of the extra overhead of \emph{INT} instruction.  \pause
        \begin{itemize}
        \item Only supported on processors after Pentium II, i.e., Family 6, Model 3, Stepping 3.  \pause
        \end{itemize}
      \item ARM processors use \emph{swi  \footnote{Short for SoftWare
        Interrupt.} to trap into the operating system.}
      \end{itemize}
    \end{itemize}
  \end{frame}

  %% PAGE
  \begin{frame}
    \frametitle{System call v.s. library function} \pause
    \begin{itemize}
    \item System call will trap into the OS kernel; while library function does not.  \pause
      \begin{itemize}
      \item So, system call is MUCH more slow than library function.  \pause
      \end{itemize}
    \item Library function is the same as the user-defined function. We can replace an existing library function with our own versions, but we can't replace a system call.  \pause
    \item A system call in one operating system may become a library function in another operating system and vice versa.  \pause
    \end{itemize}
    \begin{center}
      \includegraphics[scale=.4]{syscallvslibcall}
    \end{center}
  \end{frame}

  %% PAGE
  \begin{frame}
    \frametitle{Questions}
    \begin{itemize}
    \item Any questions? 
    \end{itemize}
    \begin{center}
      \includegraphics[scale=.5]{question}
    \end{center}
  \end{frame}

  \subsection{Function calling convention in C}

  %% PAGE
  \begin{frame}[fragile]
    \frametitle{Function calling convention in C} \pause
    \begin{minipage}[c]{0.4\textwidth}
\begin{lstlisting}
int foo(int a, int b)
{
     char buf[4];
     return a+b;
}
int main()
{
     int i;
     i = foo(12, 34);
     return i;
}
\end{lstlisting}
      \pause
    \end{minipage}%
    \begin{minipage}[c]{0.6\textwidth}
      \setlength{\unitlength}{0.5cm}
      \begin{picture}(10, 10)
        \put(4, 0){\line(0, 1){10}}
        \put(8, 0){\line(0, 1){10}}
        \put(8.5, 0){Low addr.}
        \put(8.5, 9.5){High addr.}
        \put(5, 2){\vector(-1, 0){1}}
        \put(5.5, 1.8){4B}
        \put(7, 2){\vector(1, 0){1}}
        \put(5, 0.1){Stack}

        \pause

        \put(4, 8){\framebox(4, 1){34}} \pause
        \put(4, 7){\framebox(4, 1){12}} \pause
        \put(4, 6){\framebox(4, 1){return addr.}} \pause
        \put(4, 5){\framebox(4, 1){old fp}}\pause
        \put(2.3, 5.4){fp} \pause
        \put(4, 4){\framebox(1, 1){[3]}}
        \put(5, 4){\framebox(1, 1){[2]}}
        \put(6, 4){\framebox(1, 1){[1]}}
        \put(7, 4){\framebox(1, 1){[0]}} \pause
        \put(8, 4.4){ buf} \pause
        \put(4, 3){\framebox(4, 1){......}} \pause
        \put(2.6, 6.4){+4 } \pause
        \put(2.6, 7.4){+8 } \pause
        \put(2.6, 8.4){+c } \pause
        \put(2.8, 4.4){-4 } \pause

        \color{red}\put(3.9, 2.9){\framebox(4.2, 6.2){}}\color{black} \pause
      \end{picture}
      \begin{itemize}
      \item The area enclosed by red rectangle is called \textbf{stack frame}, or simply \textbf{frame}.  \pause
        \begin{itemize}
        \item Stack frames are chained into a singly-linked list via \textbf{fp} (short for \textbf{frame pointer}). 
        \end{itemize}
      \end{itemize}
    \end{minipage}
\end{frame}

  %% PAGE
  \begin{frame}
    \frametitle{Stack frame} \pause
    \begin{itemize}
    \item The stack frame (also known as \textbf{activation record}) is used to record the information of function calling.  \pause
      \begin{itemize}
      \item Parameters  \pause
      \item Return address  \pause
      \item Local variables allocation and  \pause
      \item a ``prev'' pointer to previous stack frame.  \pause
      \end{itemize}
    \item It's stored in the \textbf{stack} of the running program. 
    \end{itemize}
  \end{frame}

  %% PAGE
  \begin{frame}
    \frametitle{Questions}
    \begin{itemize}
    \item Any questions? 
    \end{itemize}
    \begin{center}
      \includegraphics[scale=.5]{question}
    \end{center}
  \end{frame}

  %% PAGE
\ifxetexorluatex\else
\end{CJK*}
\fi
\end{document}
